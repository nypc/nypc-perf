\documentclass{article}
\usepackage{graphicx} % Required for inserting images
\usepackage{kotex}
\usepackage[style=numeric]{biblatex}
\usepackage{amsmath}
\addbibresource{references.bib} % use the name of your .bib file

\newcommand{\Prf}[1]{\Pr(#1)}

\title{NYPC Code Battle 퍼포먼스 시스템}
\author{NEXON Korea}
\date{July 2025}

\begin{document}

\maketitle

\section{퍼포먼스 모델}

대부분의 레이팅 시스템은 실력의 변동을 가정하지만 제출된 하나의 코드에 대한 실력의 변동은 없다고 가정하는게 더 맞습니다. 이를 위해서 Code Battle의 퍼포먼스 시스템은 변형된 Bradley-Terry Model을 사용합니다. Bradley-Terry Model은 각자에게 고정된 퍼포먼스 $\pi_i$를 할당하고, $i$가 $j$를 이길 확률 $\Prf{i>j}$를 다음과 같이 가정합니다. \cite{azermelo1929}

$$\Prf{i > j} = \frac{\pi_i}{\pi_i + \pi_j} $$

\subsection{샘플 AI와의 퍼포먼스}

샘플 AI와의 퍼포먼스는 참가자의 대략적인 퍼포먼스 수준을 계산하기 위해 도입했습니다. 

샘플 AI와 승리 횟수를 $w$, 패배 횟수를 $l$이라고 할 때, 승리 확률을 $\frac{2w+1}{2w+2l+2}$ 으로 추정합니다. 여기서 샘플 AI의 퍼포먼스를 $1$이라고 하면 유저의 퍼포먼스는 $\pi_i = \frac{2w+1}{2l+1}$이 됩니다.

\subsection{중간평가의 퍼포먼스}

우리는 한 가지의 가정 $\log (\pi_i) \sim Z$를 더 추가해서, 참가자들의 실력을 정규분포에 맞게 배치하려고 합니다.


$w_{ij}$를 $i$가 $j$를 이긴 횟수라고 할때, log-likelihood는 다음과 같습니다.

\begin{align*}
l(\vec{\pi}) 
&= \log\left(
    \prod_{i,j} \left[\Prf{i > j}\right]^{w_{ij}} 
    \times \prod_i \exp\left(-\frac{(\log \pi_i)^2}{2}\right)
  \right) \\
&= \sum_{i,j} w_{ij} \log\left(\frac{\pi_i}{\pi_i + \pi_j}\right)
  + \sum_i \left(-\frac{(\log \pi_i)^2}{2}\right) \\
&= \sum_{i,j} w_{ij} \left(\log(\pi_i) - \log(\pi_i + \pi_j)\right) - \sum_i \frac{\log(\pi_i)^2}{2} 
\end{align*}

이 $l(\vec{\pi})$의 maximum을 구하기 위해서 $\pi_i$에 대해서 미분을 한 후, \cite{jmlr}의 방법을 이용하여 식을 정리합니다.


\begin{align*}
\frac{\partial l(\vec{\pi})}{\partial \pi_i} &= \sum_j \left(\frac{w_{ij}}{\pi_i} - \frac{w_{ij} + w_{ji}}{\pi_i+\pi_j}\right) - \frac{\log \pi_i}{\pi_i} \\
&= \sum_j \left(\frac{\pi_j w_{ij}}{\pi_i (\pi_i + \pi_j)} - \frac{w_{ji}}{\pi_i + \pi_j} \right) - \frac{\log \pi_i}{\pi_i} \\
&= \frac{1}{\pi_i} \left(\sum_j\frac{\pi_jw_{ij}}{\pi_i + \pi_j} - \log \pi_i\right) - \sum_j \frac{w_{ji}}{\pi_i + \pi_j} = 0
\end{align*}

이 해를 직접 구하는 것은 어려운 일이므로, 적당한 방법으로 해를 근사해야합니다.

\subsubsection{계산}

이제, 양변에 $\pi_i$를 곱한 이후 이항을 해 $\log \pi_i$에 대해 정리하면

\begin{align*}
\log{\pi_i} = \sum_j \frac{\pi_j w_{ij} - \pi_i w_{ji}}{\pi_i + \pi_j}
\end{align*}

가 됩니다.


$\beta_i = \log{\pi_i}$로 정의합니다. $\beta_i$에 대해 식을 정리하면

\begin{align*}
\beta_i = \sum_j \frac{\left(e^{\beta_j-\beta_i} \cdot w_{ij}\right) - w_{ji}}{1 + e^{\beta_j-\beta_i}}
\end{align*}

가 됩니다. $\beta_{j \ne i}$가 고정되었을 때, $\beta_i$는 다음 증가함수 $f$ 의 유일한 근입니다.

\begin{align*}
g(\lambda) & =  \sum_j \frac{\left(e^{\beta_j-\lambda} \cdot w_{ij}\right) - w_{ji}}{1 + e^{\beta_j-\lambda}} \\
f(\lambda) & = \lambda - g(\lambda)
\end{align*}


이 해를 뉴턴-랩슨법으로 찾기 위해 $g'(\lambda)$를 계산합시다. $g(\lambda)$가 $e^{\beta_j - \lambda}$에 관한 함수의 합이기 때문에, $x_j=e^{\beta_j - \lambda}$라고 놓으면

\begin{align*}
\frac{dg}{ d\lambda} & = \sum_j\left[\frac{\partial g}{\partial x_j}\frac{\partial x_j}{\partial\lambda}\right] \\
& = \sum_j \left[\frac\partial {\partial x_j} \left(\frac{x_j w_{ij} - w_{ji}}{1 + x_j}\right) \cdot -x_j \right] \\
& = - \sum_j \left[\frac{ (w_{ij} + w_{ji})x_j}{(1+x_j)^2} \right]
\end{align*}

이고

\begin{align*}
\frac{df}{d \lambda} & = 1 + \sum_j\left[\frac{e^{\beta_j - \lambda} \cdot (w_{ij} + w_{ji})}{(1+e^{\beta_j - \lambda})^2}\right]    
\end{align*}

입니다.

이제 새로운 $\beta_i' $를 $\beta_i - \frac{f(\beta_i)}{f'(\beta_i)}$로 대체하는 것을 원하는 정밀도가 될 때까지 반복합니다. 

\section{퍼포먼스 표시}
계산이 끝난 이후 $\pi_i$의 값에 대해 $p_i = 1500 + 400 \pi_i$를 계산하고, $P_i$를 $50$의 배수 (샘플 AI와의 퍼포먼스) 혹은 $10$의 배수 (중간평가의 퍼포먼스) 단위로 보여줍니다. 

\begin{equation*}
P_i =
\begin{cases}
300 \times \exp((p_i-300)/400) & \text{if  } p_i < 300 \\
p_i & \text{if  } 300 \le p_i \le 2700 \\
2700 + 400\log ( 1 + (p_i-2700)/400) & \text{if  } p_i > 2700 
\end{cases}
\end{equation*}

\printbibliography






\end{document}
